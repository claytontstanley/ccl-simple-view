%%%%%%%%%%%%%%%%%%%%%%%%%%%%%%%%%%%%%%%%%
% Simple Sectioned Essay Template
% LaTeX Template
%
% This template has been downloaded from:
% http://www.latextemplates.com
%
% Note:
% The \lipsum[#] commands throughout this template generate dummy text
% to fill the template out. These commands should all be removed when 
% writing essay content.
%
%%%%%%%%%%%%%%%%%%%%%%%%%%%%%%%%%%%%%%%%%

%----------------------------------------------------------------------------------------
%	PACKAGES AND OTHER DOCUMENT CONFIGURATIONS
%----------------------------------------------------------------------------------------

\documentclass[12pt]{article} % Default font size is 12pt, it can be changed here

\usepackage{geometry} % Required to change the page size to A4
\geometry{a4paper} % Set the page size to be A4 as opposed to the default US Letter

\usepackage{graphicx} % Required for including pictures

\usepackage{enumitem}
\setlist{nolistsep} % or \setlist{noitemsep} to leave space around whole list
\usepackage{float} % Allows putting an [H] in \ldots to specify the exact location of the figure
\usepackage{wrapfig} % Allows in-line images such as the example fish picture

\usepackage{lipsum} % Used for inserting dummy 'Lorem ipsum' text into the template
\usepackage{hyperref}

\linespread{1.2} % Line spacing

%\setlength\parindent{0pt} % Uncomment to remove all indentation from paragraphs

\graphicspath{{./Pictures/}} % Specifies the directory where pictures are stored

\newcommand{\code}[1]{\texttt{#1}}
\newcommand{\filesys}[1]{\texttt{#1}}

\begin{document}

%----------------------------------------------------------------------------------------
%	TITLE PAGE
%----------------------------------------------------------------------------------------

\begin{titlepage}

  \newcommand{\HRule}{\rule{\linewidth}{0.5mm}} % Defines a new command for the horizontal lines, change thickness here

  \center % Center everything on the page

  \textsc{\LARGE Rice University}\\[1.5cm] % Name of your university/college
  \textsc{\Large Department of Psychology}\\[0.5cm] % Major heading such as course name
  \textsc{\large Computer Human Interaction Laboratory (CHIL)}\\[0.5cm] % Minor heading such as course title


  \HRule \\[0.4cm]
  { \huge \bfseries Lab Transition From MCL to Clozure Common Lisp}\\[0.4cm] % Title of your document
  \HRule \\[1.5cm]

  \begin{minipage}{0.4\textwidth}
    \begin{flushleft} \large
      \emph{Author:}\\
      Clayton Stanley % Your name
    \end{flushleft}
  \end{minipage}
  ~
  \begin{minipage}{0.4\textwidth}
    \begin{flushright} \large
      \emph{Supervisor:} \\
      Dr. Mike Byrne % Supervisor's Name
    \end{flushright}
  \end{minipage}\\[4cm]

  {\large \today}\\[1cm] % Date, change the \today to a set date if you want to be precise

  \includegraphics[scale=.5]{ChilLogo} % Include a department/university logo - this will require the graphicx package

  \vfill % Fill the rest of the page with whitespace

\end{titlepage}

%----------------------------------------------------------------------------------------
%	TABLE OF CONTENTS
%----------------------------------------------------------------------------------------

\tableofcontents % Include a table of contents

\newpage % Begins the essay on a new page instead of on the same page as the table of contents 

%----------------------------------------------------------------------------------------
%	INTRODUCTION
%----------------------------------------------------------------------------------------

\section{Introduction} % Major section

The purpose of this code is to enable Clozure Common Lisp to read in GUI source code written for Macintosh Common Lisp.
This enables task environments written in MCL (e.g., Phaser, Votebox, NextGen) to work with CCL with minimal code modifications. 

\subsection{Design Notes}

Currently, three files are generated during the build to achieve this purpose:

\begin{enumerate}[topsep=12pt]
  \item \filesys{ccl-simple-view.lisp}: \filesys{submodules/actr6/support}
  \item \filesys{device.lisp}: \filesys{submodules/actr6/devices/ccl/device.lisp}
  \item \filesys{uwi.lisp}: \filesys{submodules/actr6/devices/ccl/uwi.lisp}
\end{enumerate}

The file \filesys{ccl-simple-view.lisp} provides the language layer for CCL, so that CCL can read in MCL GUI source code.
You'll see a subset of Digitool's GUI specification implemented for CCL in this file.

Files \filesys{device.lisp} and \filesys{uwi.lisp} provide the ACT-R device for CCL, so that models that interact with the task environments written for MCL can now work with CCL.
These files are essentially copies of the analogous files for MCL except at the top they require \filesys{ccl-simple-view.lisp}.
This is so the MCL language layer is available in CCL before the rest of the code is read in.

File \filesys{ccl-simple-view.lisp} is a concatenation of many smaller .lisp files in this repository.
This was done because I find it easier / more productive to work with many smaller files during development, but it's easier / more straightforward to provide a single file that does a single purpose to the user. 

File \filesys{ccl-simple-view.lisp} implements Digitool's GUI specification by leveraging Clozure's Objective C bridge and Apple's Cocoa framework.
The majority of the structure of this file is written in CLOS style, and is built on top of Clozure's provided ``easygui'' package. 

\section{Preparing Repository for Development}

In order to run the unit and regression tests, and work on the codebase, there are a few dependencies that must be installed first.
To check that your system has these dependencies installed, run \code{./check-dev-env.sh}.
This file will notify you if any dependencies need to be installed, and will offer suggestions on how to install them.
During this process, all submodules for the repository will be initialized and updated.
Once you can run this file without error, then your development environment is correctly configured.

%----------------------------------------------------------------------------------------
%	MAJOR SECTION 1
%----------------------------------------------------------------------------------------
\section{Running the Tests}

Tests can be run in two ways: From the command line using make targets, or directly within your CCL development REPL.
The only tests that should be run from the command line are the regression and unit tests.
It's not recommended to run a test for a single file from the command line.
The command line targets are used during the deployment process, to ensure that the recent commit is valid.
And I've used some optimization techniques in this case.
Specifically, before running these tests, a standalone image of the current source in the repository is created, and then each test is executed by first loading that standalone image.
This saves time by not recompiling the CCL Cocoa libraries for each test, and also isolates each test from each other by starting with a fresh CCL image before running each test.

However these optimization decisions mean that if you were to run a single test from the command line, it would use the previously-built CCL image when running that test.
If you were to change a line of source code and then run the test, the CCL image would not contain that source code change.
So if you're in the middle of development and want to run a single test, and ensure that the test loads the most recent source, run the test from within your development REPL.

\subsection{Regression and Unit Tests}

After the development environment has been configured, you can run the unit tests by typing \code{make -C testing unit} at the top level of the repository.
This will execute make in the testing folder, and tell it to fire the unit target, which will run all unit tests.
If you want to run the more lengthy and thorough regression tests, use the \code{continuous-integration} target instead of the \code{unit} target.

\subsection{Individual Tests}

Each file in the \filesys{./testing} directory is an individual test that can be run in isolation (as opposed to running during the unit or continuous-integration tests).
The most straightforward way to run a single test file is to launch your CCL development environment and load the file.

Note that the current bootstrap code for the tests assumes that a development environment is a ``SLIME'' session (using either SLIME or slimv).
So if you are using a different development environment (e.g., working from the included CCL GUI), then the \filesys{bootstrap.lisp} file will not work properly.
One current workaround is to add \code{swank-repl} to \code{*modules*} before loading a test.
You could place this in \code{\textasciitilde/.ccl-init.lisp}, or (my recommendation) use a slime implementation, or
(better fix if you don't want to) change the logic in the \filesys{bootstrap.lisp} file so that it does a better check to figure out if you're currently in a dev REPL or launched from the command line.

\subsection{The \filesys{testing/bootstrap.lisp} File}

Each test first loads a bootstrap file.
That file will determine how to load the ccl-simple-view source and then load it, among other things.
The main branch in that file is related to how ccl-simple-view gets loaded.
If the test was launched from the command line, then the deployed \filesys{ccl-simple-view.lisp} will get loaded.
If the test was launched from a CCL development REPL, then each source file in the repository that makes up \filesys{ccl-simple-view.lisp} will get loaded directly.

\section{Making a Build}

Once you have made a new commit and are ready to make a build, execute \code{make zip-actr6} from the top level directory.
This creates the three build files from the current source, and then places a zip file in the actr6 submodules top level directory.
This file is what I'm currently emailing to Dan whenever I'm ready to place a new build in the ACT-R distribution.

Other builds can be made as well, like building a Votebox test for Kristen, or a Phaser test for Frank, or a NextGen test for Jeff.
The targets for these builds are \code{zip-Votebox}, \code{zip-tamborello}, and \code{zip-NextGen} respectively.
Executing each target will make a zip file with all necessary source code and dependencies needed to run the test, and place that zip file at the top level directory.
Within that zip file in the \filesys{testing} folder, will be the test file that is associated with each of builds. 
Another developer without even checking out the repository source code should be able to unzip that file, load the Clozure App, load ACT-R, and then load that test file.

All of these builds rely on properly documenting dependencies for each of them.
For example, the Phaser tasks depend on a few files in the \filesys{bincarbon} folder (like \filesys{procedure-window2.lisp}),
and the Votebox tasks need to load the Votebox emulator before loading the model files.
All of this is taken care of within the \filesys{testing/file-lists} directory.
Each test file in testing has an associated file-list file in \filesys{testing/file-lists}, which lists each file that must be loaded in order to run the test.
Those files in \filesys{file-list} are examined when making builds as well, so that all dependencies are included in the builds.
A lot of time and care has been put into creating these build dependency files.
Even with all the simplifying that I've done, the build scripts are probably the most complicated part of this entire development environment (as is usually the case with build scripts).
So I'd recommend taking the time to digest how the top-level \filesys{Makefile}, the \filesys{testing/bootstrap.lisp} code, the \filesys{testing/file-lists} files, and the \filesys{build/Makefile}, all interact.

\section{Deploying Code Changes}

You've made a new commit and would like to deploy the current code in your development repository to RedGiant.
This process is done in two steps: Push your changes and run the deploy target.
To push your changes to \filesys{chil.rice.edu}, do a \code{git push}.
To run the deploy target, execute \code{make -C build deploy} from the top-level directory.
This deploy script will update all folders on RedGiant that are associated with the repository.
This includes the Projects/mcl-migration folder, and also all submodules that have a location on RedGiant (e.g., bincarbon, NextGen, Phaser, etc.).
The list of deployed locations is included in the \filesys{builds} directory.

\section{Looking at Code Coverage}

An html code coverage report for ccl-simple-view.lisp is generated on the deployed machine after pushing the code changes.
The \filesys{report.html} file is located in the \filesys{coverage} directory on the deployed machine.

\section{Directory Structure} % Major section

\begin{itemize}
  \item actr6: All .lisp files needed to build an ACT-R device for CCL that uses a Cocoa display. Note the share.lisp file lives here for now, which contains the MCL GUI interface for CCL
  \item \filesys{bincarbon}:
    All original MCL bincarbon code as well as CCL-specific utility files. 
    Most original MCL files have been ported to work with both CCL and MCL.
    Timer is broken on CCL currently though.
    Other files, like CFBundle.lisp don't work at all on CCL, but aren't necessary on CCL.
    This folder is itself a git repository, so it is added as a submodule for this repository.
  \item \filesys{build}: Code to generate a single source file from current code in the repository.
  \item \filesys{docs}:
    Archived documentation and reference manuals found online.
    Also contains source to build this document.
  \item \filesys{easygui}:
    Any .lisp extensions to CCL's easygui package.
    Contains a few bug fixes, and an extension which provides a Cocoa view that does not respond to mouse activity.
  \item \filesys{fonts}: Contains font dependencies used in some of the testing source code (e.g., Tamborello's Phaser experiments).
  \item \filesys{rmcl}:
    Any lisp code from RMCL that was either directly copied to this distribution (some of the Digitool GUI code could be bootstrapped after the CCL interface was defined)
    or rewritten for this distribution (e.g., the modal dialog implementation)
  \item \filesys{spell}: Contains lisp of words that vim ignores when spell checking repository.
  \item \filesys{submodules}: Source code for actr and ccl
    \begin{itemize}
      \item \filesys{actr} is newest version (as of Jun 2013). All code is same except that loader files (loader.lisp) are inside each of the tutorial folders
      \item \filesys{ccl-1.9} is newest version (as of Jun 2013). Original image. Used to reference ccl source and run basic ccl core file
      \item \filesys{lisp-dev} is a git repository for building a custom ccl core that I use during development.
	I'm using some of the source for the utilities in the GUI code, so the builds here grab some of that source.
      \item \filesys{rmcl} is the newest version; no changes; used to reference rmcl source, and run rmcl for each of the tests.
    \end{itemize}
  \item \filesys{testing}: Source code for testing the ccl code
    \begin{itemize}
      \item \filesys{testCCLDevice.lisp}: Used to test basic functionality of building a Cocoa display by writing lisp code that meets the uwi.lisp interface spec
      \item \filesys{testTutorials.lisp}: Runs ACT-R through all tutorials; checks that ACT-R can 'see' CCL's Cocoa device
      \item \filesys{testVotebox.lisp}: Tests to run Votebox on CCL
      \item \filesys{testTamboDisModel.lisp}: Tests to run Phaser on CCL
    \end{itemize}
  \item tools: Various shell scripts and tools used to manage the repository.
    \begin{itemize}
      \item \filesys{merge-and-verify-driver}. Used by git when I want to do a merge by hand.
    \end{itemize}
\end{itemize}

\end{document}

